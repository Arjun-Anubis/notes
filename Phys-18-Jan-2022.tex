% Created 2022-01-20 Thu 11:22
% Intended LaTeX compiler: pdflatex
\documentclass[11pt]{article}
\usepackage[latin1]{inputenc}
\usepackage[T1]{fontenc}
\usepackage{graphicx}
\usepackage{grffile}
\usepackage{longtable}
\usepackage{wrapfig}
\usepackage{rotating}
\usepackage[normalem]{ulem}
\usepackage{amsmath}
\usepackage{textcomp}
\usepackage{amssymb}
\usepackage{capt-of}
\usepackage{hyperref}
\author{anubi}
\date{\today}
\title{}
\hypersetup{
 pdfauthor={anubi},
 pdftitle={},
 pdfkeywords={},
 pdfsubject={},
 pdfcreator={Emacs 27.1 (Org mode 9.3)}, 
 pdflang={English}}
\begin{document}

\tableofcontents

\section{Syllabus}
\label{sec:org8d5c360}
\subsection{Physics}
\label{sec:org00fbd44}
\subsubsection{Motion in one dimension}
\label{sec:orge0fb752}
\subsubsection{Laws of motion}
\label{sec:org0e6c076}
\subsubsection{Gravitation}
\label{sec:org4ebb276}
\subsubsection{Heat}
\label{sec:orgfaa3b0b}
\begin{enumerate}
\item Definition of temperature
\label{sec:org93de318}

\begin{itemize}
\item \emph{Degree of hottness or coldness of an object.}
\item \emph{Average total kinetic energy in an object}
\end{itemize}
\emph{- It determines the direction of flow of heat}

Unit - K, C
0\textdegree{} C = 273k
\item Definition of Heat
\label{sec:org1100ba4}

\emph{- Sum total energy of all the molecules in a substance}

\texttt{AKA:} \textbf{Thermal Energy}, \textbf{Heat Energy}

Unit - Joule, Watt-Hour, cal
1 cal = 4.186 Joule

\item Thermal equilibrium
\label{sec:org9d55ff3}

\begin{center}
\includegraphics[width=.9\linewidth]{Physics/2022-01-18_13-59-40_screenshot.png}
\end{center}

\begin{itemize}
\item The condition when objects in contact with each other do not further exchange head it called \emph{Thermal Equilibrium}
\end{itemize}

\item Difference between heat and temperature
\label{sec:org59a0828}
\begin{center}
\begin{tabular}{rll}
\hline
No. & Heat & Temperature\\
\hline
1. & Heat is a form of energy & Temperature is a degree of hotness or colness\\
2. & Heat is the cause of temperature & Temperature is an effect of heat\\
3. & - & Temperature determines the direction of heat.\\
4. & Unit - joule & Unit - kelvin\\
\hline
\end{tabular}
\end{center}
\item Anamolous Expansion of Water
\label{sec:orgc12ef5f}

\begin{center}
\includegraphics[width=.9\linewidth]{Heat/2022-01-18_13-27-22_screenshot.png}
\end{center} 
\begin{itemize}
\item \emph{The expansion of water when it is cooled below 4\textdegree{} C}
\end{itemize}

The Density of water is minimum at the temperature of 4\textdegree{} C at 1
gm/cc. Which is the standard density of water given in most places.

\begin{enumerate}
\item Pratical Applications
\label{sec:orga8d67db}

\begin{enumerate}
\item Survival of Aquatic Creatures
\label{sec:orgcf24a12}
In Northern climates lakes would fully freeze over, rather than just
the surface because the surface would be the last to freeze anyway,
but because at 4C the water sinks. Meaning at 4C The lake would be
uniformly heated.

\item Water Pipes burst
\label{sec:orga8801eb}
The expansion of water causes the pipes to burst
\item Fruits Burst
\label{sec:orge228896}
The water present in each of the cells expand causing great pressure
\item Frost bite
\label{sec:orgc1d1027}
Similarly, for humans
\end{enumerate}
\item {\bfseries\sffamily TODO} Q1 Q7
\label{sec:org5be6ca2}
\end{enumerate}
\end{enumerate}
\subsubsection{Energy}
\label{sec:org17eb3da}
\subsubsection{Reflection of light}
\label{sec:org1352a4c}
\begin{enumerate}
\item The nature of light
\label{sec:orgbf6142d}
\begin{itemize}
\item Light is a form of energy.
\item Wave nature, particular nature, transverse
\end{itemize}
wave ( transverse - perpendicular ), ( logngitudinal is along the
length ).
\begin{enumerate}
\item {\bfseries\sffamily TODO} Speed of light 3 * 10\textsuperscript{8} m/s ( by definition,  exact. 299,792,458 )
\label{sec:org1dcdd58}
In air, same as vaccum, in glass 2, and in water 2.25
\end{enumerate}

\item Reflection of light
\label{sec:orgcf30cf8}
\begin{itemize}
\item /Reflection is the phenomenon in which rays of light on striking a
\end{itemize}
surface are sent back into the same medium/
\item Definitions of Some terms
\label{sec:org82beeba}
\begin{enumerate}
\item Points
\label{sec:org8c36c61}
\begin{enumerate}
\item Point of incidence
Point where the ray strikes the mirror
\item Point of reflection
Point from which the ray emerges
\end{enumerate}
\item Rays
\label{sec:org059d89f}
\begin{enumerate}
\item Incident ray
The ray that strickes the mirror's reflecting surface
\item Reflected ray
The ray that emerges from the mirror
\item Normal
A line drawn perpendicular to the mirror at the point of incidence
\end{enumerate}
\item Angles
\label{sec:org32e3cec}
\begin{enumerate}
\item Angle of incidence
Angle between the incident ray and normal
\item Angle of reflection
Angle between the reflected ray and normal
\item Glance angle of incidence
Angle between the incident ray and mirror
\item Glance angle of reflection
Angle between the reflected ray and mirror
\end{enumerate}
\end{enumerate}
\item Laws of Reflection
\label{sec:orgce84a67}
\begin{enumerate}
\item The incident ray, the normal, and the reflected ray are all in the same plance
\item The angle of incidence is equal to the angle if reflection
\end{enumerate}
\item Types of Images
\label{sec:org31dc3c8}
\begin{enumerate}
\item Real Image
\label{sec:orgbb03a83}
Can be obtained on a screen. Is inverted. Example cinema screen
\item Virtual Image
\label{sec:org833bc7d}
Can \uline{not} be obtained on a screen. Is errect. Example plane mirror.
\end{enumerate}
\item Differences
\label{sec:orgee87c5f}
\begin{center}
\begin{tabular}{lll}
Parameter & Real & Virtual\\
\hline
Formation & It is fromed when two rays inersect at a point in front of the mirror & When two or more rays appear to meet behind a screen\\
Screen & It can be obtained on a screen & It can \uline{not}\\
Nature & It is always inverted & It is alwyas \uline{errect}\\
\end{tabular}
\end{center}

\item Pair of mirros formula
\label{sec:orgdc54121}
\item Parallel mirrors
\label{sec:orgfa4b16c}
\item Uses of plane mirrors
\label{sec:org74dfe69}
\end{enumerate}
\subsubsection{Sound}
\label{sec:org72af136}
\subsubsection{Electricty}
\label{sec:orgfdd33a6}
\begin{enumerate}
\item Electricty
\label{sec:orga1ff7bf}
\end{enumerate}
\subsubsection{Magnetism}
\label{sec:org2c01cd7}
\begin{enumerate}
\item Natural Magnet
\label{sec:org753b34a}
\item Magnet
\label{sec:orge2e7a0d}
\item Magnetic Poles
\label{sec:org5448fbb}
\item Magnetic and Non-Magnetic substances
\label{sec:orgde54587}
\emph{Substances that are attracted to magnets are called magnetic}
\item Magnetic Compass
\label{sec:org4e9a052}
\begin{itemize}
\item \emph{It is an instrument consisting of a suspended magnet which can be used for navigation}
\end{itemize}
\item Properties of Magnets
\label{sec:orgbb054b7}
\begin{enumerate}
\item Attractive Property
\label{sec:org3ffae4a}
\item Directive Property
\label{sec:org137a0a6}
\item Attraction and Replusion
\label{sec:orgedb7320}
\item No such thing as a UniPole
\label{sec:org0f30895}
\end{enumerate}
\item Induced Magnetism
\label{sec:org4564a64}
\item Earth's Magnetic Field
\label{sec:orgbcbd9f6}
\item Uniform and Non Uniform Magnetic Fields
\label{sec:orgef4c4fb}
Uniform - Center of bar magnet, lines are parallel
Non Uniform - Everything else
\item MLOF
\label{sec:orgbf586cb}
\item Null Points
\label{sec:orgc8c9c62}
Points where the value of the magnetic field is 0.
\item Electro-Magnet
\label{sec:org21c4303}
\begin{itemize}
\item \emph{A manget whose magnetic field is induced by electric current}
\end{itemize}
\item Uses of Magnets
\label{sec:orgbcc8616}
\item Electric Bell
\label{sec:orgd080c96}
It works on the prinnciple of electro magnetism.

\begin{itemize}
\item On pressing the key the current starts flowing in the circuit of the
\end{itemize}
electric bell.

\begin{itemize}
\item When the current passes through the electromegnet it
\end{itemize}
attracts the hammer towards itself which strikes the gong

\begin{itemize}
\item When the hammer touches the gong the circuit breaks down. Current
\end{itemize}
stops flowing, due to which the electromagent loses its
electromgentism and returns to its orignial positionn

\begin{itemize}
\item When the hammer comes back the setup is returned to its original position
\end{itemize}
\end{enumerate}

\subsection{English}
\label{sec:orgfe47b99}
\subsubsection{Merchant of Venice}
\label{sec:orge230fe3}
\begin{enumerate}
\item {\bfseries\sffamily TODO} Worksheets
\label{sec:orgb176a0c}

\begin{center}
\includegraphics[width=.9\linewidth]{English/2022-01-19_09-14-35_Eng_IX_-Lit-Worksheet 9- MOV- Act 1 Sc 1.pdf}
\end{center}


\begin{center}
\includegraphics[width=.9\linewidth]{English/2022-01-19_09-14-54_Eng_IX_-Lit-Worksheet 9-MOV I (ii).pdf}
\end{center}

\begin{center}
\includegraphics[width=.9\linewidth]{English/2022-01-19_09-15-02_Eng_Lit_MOV_Worksheet 10.pdf}
\end{center}

\item Act 1
\label{sec:org0dd0ec8}
\begin{enumerate}
\item Scene 1
\label{sec:org47c2b09}
Whole discussion about why antonio is sad, followed by bassanio
revealing to antonio what he had wanted to talk about

\begin{enumerate}
\item Segment 1
\label{sec:org6645f50}
Antonio \(\rightarrow\) Salarino, Salanio: Reasons why antonio is sad
\item Segment 2
\label{sec:orgee23619}
Antonio \(\rightarrow\) Gratiano, Lorenzo: Gratiano not letting lorenzo speak
\item Segment 3
\label{sec:orgf740319}
Antonio \(\rightarrow\) Bassanio: Portia


\item References
\label{sec:org4f88af4}
\begin{enumerate}
\item Middle Age References
\label{sec:orgb49ec70}
\begin{itemize}
\item Jaundice
\item Livers being the most important organ in the body
\end{itemize}
\item Greek References
\label{sec:org0b969d2}
\begin{itemize}
\item Nestor ( Greek general, never laughed )
\item Jason
\item Oracle
\item Golden fleece
\end{itemize}
\item Roman References
\label{sec:orgf095059}
\begin{itemize}
\item Janus
\item Portia (the other one)
\item Brutus
\item Cato
\end{itemize}
\end{enumerate}
\end{enumerate}
\item Scene 2
\label{sec:org93a9161}
Portia and nerissa discusss portia's suitor. First she cribs a little
bit about her father's rules and then it's made clear she likes
bassanio
\begin{enumerate}
\item Cribbing
\label{sec:orgbabba18}
If it do was as easy as to know what were good to do, Chapels had
been chruches as poor men's cottages princes' palaces. It is a good
divine that follows his own instructions. I can easier teach twenty
what were good to be done, that ne one of the twenty to follow mine
own teaching. The brain may devise laws for blood, but a hot temper
leaps o'er a cold decree: such a hare is madness of youth, to skit
o'er the meshes of good cousel, the cripple
\item Neapolitan Prince
\label{sec:orgcce6f05}
From naples, Italy
(map)

Greatest Achievement: Shoeing his horse

Hobby: Talking about how he can shoe his horse.

\item County palatine
\label{sec:org9b6af1a}
South Germany ( does not exist yet ), so palatinate

Future expectations: Going to live in the mountains abandoning humaity (\texttt{Heraclitus of Ephesus})

Hobbies: Sulking

\emph{I had rather be married to death's head with a bone in his mouth}
\item Monsieur Le bon
\label{sec:org1904a09}
France

Stunning prince from france, beats down both County palatine \emph{and}
Neaplitan\ldots{}. In being a horrible husband. He has the temprement of
Count plataine. And the boastfulnes of the Neapolitan prince.

\emph{If he would despise me, I should forgive him. for if he love me to
maddness, I shall never requite him}
\item Baron Falconbridge
\label{sec:org64123a4}
From, England

Languages: \emph{pas de} \emph{Fran�ais}, \emph{non latin}, \emph{nessun italiano}
\begin{enumerate}
\item Clothes
\label{sec:org3ff2dee}
\begin{itemize}
\item Italian Doublet(Jacket)
\item French Breeches
\item German Bonnet
\item Everywherian behaviour
\end{itemize}

\emph{alas, who can converse with a dumb show}
\end{enumerate}
\item \emph{Scottish lord}
\label{sec:orgf2950ae}
\item Nephew of Duke Saxony
\label{sec:org55ec8f0}
\item Prince Morroco
\label{sec:org1fffcb6}
\end{enumerate}
\item Scene 3
\label{sec:org9772570}
\end{enumerate}
\item Act 2
\label{sec:orgf1cbd93}
\begin{enumerate}
\item Scene 1
\label{sec:org93283ef}
\item Scene 2
\label{sec:org3cffb8d}
\item Scene 3
\label{sec:orgca562f0}
\item Scene 4
\label{sec:org8b61af2}
\item Scene 5
\label{sec:org57f2f33}
\item Scene 6
\label{sec:org74e9926}
\item Scene 7
\label{sec:orgd4d6981}
\end{enumerate}
\end{enumerate}
\subsection{Biology}
\label{sec:org25bedf6}
\subsubsection{Red Cross}
\label{sec:org5d3b786}
\begin{enumerate}
\item Origin
\label{sec:org8854224}
Originated in Geneva, Switzerland The ICRC operates in over a 100
coutries, providing relief to patients all over the world.

Jean-Henri Dunant, 1859 went to meet the french emperor napoleon the
III, to talk about difficulties in bussiness in algeria, he arrived in
the small town of Solferino. Dunant was horrified by the death he saw
at the battle of solferino, every single day 40,00 soldiers died and
for that dunant devoted himself to their aid .Before which there
weren't many aid Gustave Moynier in 1863 received denuna't book and
but it for discussion at theosciety for public welfare in geneva.

Back at home he write a book called A memory of solferino

Since this was the 19th centuary, the international committee was
attencded by the autrian empire baden, bavaria, france, hanover,
hesse, italy, netherlands, prussia, russia, saxony, spain, UK, seden,
norway
\item WW1
\label{sec:orga08a63e}
The red cross beign an international organisation does not have much
power within countries, it has to work with the red cross
organisations of various countries. 

It is stunning to imagine that they were able to muster any effort at
all considering the leading medical science had barely even accepted
thories like germ theory, they were still in the age before
anasthesia.  Practices like amputation was the de-facto response to
any disease. Aroma theory was still prevailent. The sanitation
revolution after the cholera epidemic had only recently started.

Regardless, They had to face one if the greatest disasters in world history WWI
\end{enumerate}
\subsection{Hindi}
\label{sec:org5049da5}
\subsection{Geography}
\label{sec:orgc7cfed2}
\subsubsection{Atmosphere}
\label{sec:orgae0920a}
\begin{enumerate}
\item Winds Static winds
\label{sec:orgdc36eb8}

\begin{center}
\includegraphics[width=.9\linewidth]{Winds/2022-01-18_13-05-17_screenshot.png}
\end{center}

\begin{enumerate}
\item Northern Hemisphere
\label{sec:org7963900}
\begin{enumerate}
\item North Easterlies
\label{sec:org4786510}
Polar
\item South Westerlies
\label{sec:org1df513c}
Variable
\item North-East Trade Winds
\label{sec:org9631a79}
\end{enumerate}
\item Southern Hemisphere
\label{sec:orgcf2ddb6}
\begin{enumerate}
\item South-East Trade Winds
\label{sec:org03fa555}
\item North Westerlies
\label{sec:org60bd258}
Variable, AKA: Roaring Forties, Furious Fifties, Shrieking/Gloomy Sixties
\item South Westerlies
\label{sec:orgc2eb01a}
\end{enumerate}
\end{enumerate}
\item Periodic winds
\label{sec:org633d077}
Only during a certain part of the year, or season, or day.
\begin{enumerate}
\item Daily
\label{sec:org9759893}
\begin{enumerate}
\item Sea Breeze
\label{sec:orga0d8158}
Sea -> Land
\item Land Breeze
\label{sec:org1d3deb0}
Land -> Sea
\end{enumerate}
\item Seasonal
\label{sec:orgbbd7fb2}
\begin{enumerate}
\item Monsoon
\label{sec:org459f6de}
Same cause as land and sea breeze but they are on the seasonal scale.
\begin{enumerate}
\item Summer Monsoon
\label{sec:org59bd549}
Wind blows from the Sea to the land as in summer the land mass has
lower pressure, coming from the ocean they bring rain and moisture to
the coastal region

\item Winter Monsoon
\label{sec:org58e9a54}
Much weaker, carries no water and is pretty much sporadic they winds
are cold and dry coming from central asia.
\end{enumerate}
\end{enumerate}
\end{enumerate}

\item Local Winds
\label{sec:org3491f66}
Localized only effect a small area
\begin{enumerate}
\item Loo
\label{sec:org3623796}
\item Chinook
\label{sec:orgec249ce}
\item Feohn
\label{sec:org65311f6}
\item Mistral
\label{sec:orgffe2d96}
\end{enumerate}
\item Variable Winds
\label{sec:org5e3413a}
They are caused occasionally due to the movement of pressure systems
\begin{enumerate}
\item Cyclones
\label{sec:orgf46a523}
\item Anti-Cyclones
\label{sec:org414e3c7}
\end{enumerate}
\end{enumerate}
\item Pressure Belts
\label{sec:org1d1f69f}
\begin{enumerate}
\item Low pressure eq
\label{sec:org2d1eaec}
Makes Sense its pretty hot so low pressure. This forms clouds when the
water rises and makes cloud, this also explains daily rainfall
\item High pressure Sub tropical
\label{sec:org55e327c}
Doesn't really make sense its still pretty hot. The air which has
risen from the equator, will sink when it reaches slightly higher
latitudes. This also means that there won't be any clouds because all
the water leaves as the air cools, it evaporatoes off. This means that
the sky is clear. No cloud cover means extreme temperatures.

Off hand: "\emph{This is why there are desserts there"}

\item Low Pressure Sub Polar
\label{sec:org09a69ad}
\item High Pressure Polar
\label{sec:orgede77ff}
Makes sense its prety cold so high pressure
\end{enumerate}
\item Pressure Cells
\label{sec:orgeefb1ac}
\begin{itemize}
\item Hadley cell
Between the eq and subtropical belts
\item Ferrel cell
Between the Sub Polar and and Sub tropical Belts
\item Polar cell
Between the polar and sub polar belts
\end{itemize}
\end{enumerate}
\end{document}
